We work with a two mutually dual finite-dimensional vector spaces \( N
\) and \( M \) over a linearly ordered field \( \Bbbk \). This means,
we have a bilinear pairing \( \langle \cdot, \cdot \rangle \colon N
\times M \to \Bbbk \) such that the induced maps \( M \to
\Hom(N,\Bbbk) \) and \( N \to \Hom(N,\Bbbk) \) are both bijective
(this is also known as a \emph{perfect pairing}). Since \( \Bbbk \) is
linearly ordered, we can define the non-negative scalars \(\Bbbk_{\geq
0} = \{t \in \Bbbk \mid t \geq 0 \} \subseteq \Bbbk\). This is a
\emph{semifield}, i.e. a commutative semiring, in which every nonzero
element has a multiplicative inverse.


%  todo list for blueprinting:
% done. the Minkowski-Weyl theorem
% done. faces + basic theorems (faces are polytopes, face of face is face, etc)
% done. face lattice + combinatorial equivalence + basic theorems (face lattice is graded, intervals are face lattices, etc)
% polar duality + basic theorems (polarity is an involution, flips face lattice)
% normals, normal fan, normal equivalence
% basic operations yield polytopes: product, direct sum, join, Minkowski sum, projections, sections, convex hull


\section{Polyhedral cones}

\begin{definition}
  \label{pointed-cone}
  \uses{}
  \lean{PointedCone}
  \leanok
  A {\bf pointed cone} \( \sigma \subseteq N \) is a \( \Bbbk_{\geq 0}
  \)-submodule of \( N \).
\end{definition}

\begin{definition}
  \label{cone-span}
  \uses{pointed-cone}
  \lean{PointedCone.span}
  \leanok
  For a set \( S \subseteq N \), the {\bf cone generated by \( S \)}
  is the pointed cone
  \[
      \Cone(S) := \Span_{\Bbbk_{\geq 0}}(S) \subseteq N.
  \]
\end{definition}

\begin{definition}
  \label{cone-finitely-generated}
  \uses{pointed-cone}
  \lean{Submodule.FG}
  \leanok
  A pointed cone \( \sigma \subseteq N \) is called {\bf finitely
  generated}, if \( \sigma = \Cone(S) \) for some finite set \( S
  \subseteq N \).
\end{definition}

\begin{definition}
  \label{dual-cone}
  \uses{pointed-cone}
  \lean{PointedCone.dual}
  \leanok
  For a set \( S \subseteq M \), the {\bf dual cone of \( S \)} is
  the pointed cone
  \[
      S^\vee := \{x \in N \mid \forall y \in S, \langle x, y \rangle
      \geq 0 \} \subseteq N.
  \]
\end{definition}

\begin{definition}
  \label{polyhedral-cone}
  \uses{dual-cone}
  \lean{PointedCone.IsPolyhedral}
  \leanok
  A pointed cone \( \sigma \subseteq N \) is called {\bf polyhedral},
  if \( \sigma = S^\vee \) for some finite set \( S \subseteq M \).
\end{definition}

\begin{proposition}
  \label{dual-polyhedral}
  \uses{polyhedral-cone, cone-finitely-generated}
  \lean{PointedCone.IsPolyhedral.dual_of_fg}
  \leanok
  If \( \sigma \subset N \) is finitely generated, then \( \sigma^\vee
  \subset M \) is polyhedral.
\end{proposition}
\begin{proof}
  \uses{}
  \leanok
  Let \( \sigma = \Cone(S) \) for \( S \subseteq N \). We need to show
  \( \Cone(S)^\vee = S^\vee \). The inclusion ``\( \subseteq \)'' is
  obvious, as \( S \subseteq \Cone(S) \). The opposite inclusion can
  be shown by induction on the general element of \( \Cone(S) \).
\end{proof}

\begin{proposition}[Double dual]
  \label{double-dual-polyhedral-cone}
  \uses{polyhedral-cone}
  \lean{PointedCone.IsPolyhedral.dual_dual_flip,
  PointedCone.IsPolyhedral.dual_flip_dual}
  \leanok
  If \( \sigma \subseteq N \) is polyhedral, then \( \sigma^{\vee\vee}
  = \sigma \).
\end{proposition}
\begin{proof}
  \uses{}
  \leanok
  For any set \( S \subseteq N \) it is trivial to show \(
  S^{\vee\vee\vee} = S^{\vee} \). The claim follows since \( \sigma =
  S^{\vee} \) for a finite set \( S \subseteq N \).
\end{proof}

\begin{proposition}
  \label{bot-polyhedral}
  \uses{polyhedral-cone}
  \lean{PointedCone.IsPolyhedral.bot}
  \leanok
  The zero-space \( \{0\} \subseteq N \) is a polyhedral cone.
\end{proposition}
\begin{proof}
  \uses{}
  \leanok
  If \( S \subseteq M \) is a finite \( \Bbbk \)-basis of \( M \),
  then \( \{0\} = (S \cup -S)^\vee \).
\end{proof}

\begin{proposition}[Fourier-Motkin elimination]
  \label{fourier-motzkin}
  \uses{polyhedral-cone}
  % \lean{PointedCone.dual_auxGenSet}
  % \leanok
  If \( \sigma \subseteq N \) is a polyhedral cone and \( w \in N \),
  then \( \sigma + \Cone(\{w\}) \) is polyhedral too.
\end{proposition}
\begin{proof}
  \leanok
  Write \( \sigma = S^{\vee} \) for a finite \( S \subseteq M \).
  Define
  \[
      T\ :=\ \{x \in S \mid 0 \leq \langle x, w \rangle \}\ \cup\ 
      \{y \langle x, w \rangle - x \langle y, w \rangle \mid x, y \in
      S, 0 \leq \langle x, w \rangle, \langle y, w \rangle < 0 \}\ 
      \subseteq\ M.
  \]
  Note that \( T \) is finite. We claim \( S^{\vee} + \Cone(\{w\}) =
  T^{\vee} \). Clearly \( T \subseteq \Cone(S) \), hence \( S^{\vee} =
  \Cone(S)^{\vee} \subseteq T^{\vee} \). Also clear is \( w \in
  T^{\vee} \), hence \( S^{\vee} + \Cone(\{w\}) \subset T^{\vee} \).
  For the reverse inclusion, let \( v \in T^\vee \). We need to show
  \( v - uw \in S^\vee \) for some \( u \in \Bbbk_{\geq 0} \). If
  there is no \( y \in S \) with both \( \langle y, w \rangle < 0 \)
  and \( \langle y, v \rangle < 0 \), we can easily see that \( u = 0
  \) works, i.e. \( T^{\vee} \subseteq S^{\vee} \). Otherwise, we
  set
  \[
      u\ :=\ \max_{\substack{y \in S \\ \langle y, w \rangle < 0}}
      \frac{\langle y, v \rangle}{\langle y, w \rangle},
  \]
  which is then well-defined and non-negative. We need to show
  \( u \langle z, w \rangle \leq \langle z, v \rangle \) for all \( z
  \in S \). If \( \langle z, w \rangle < 0 \), this follows from
  maximality of \( u \). If \( \langle z, w \rangle = 0 \), the claim
  follows from \( z \in T \) and \( v \in T^{\vee} \). Lastly, assume
  \( \langle z, w \rangle > 0 \) and fix \( y \in S \) such that \( u
  = \frac{\langle y, v \rangle}{\langle y, w \rangle} \). In
  particular, \( \langle y, w \rangle < 0 \). Then the claim \( u
  \langle z, w \rangle \leq \langle z, v \rangle \) is equivalent to
  \[
      \langle y, v \rangle \langle z, w \rangle \geq
      \langle z, v \rangle \langle y, w \rangle.
  \]
  But \( y \langle z, w \rangle - z \langle y, w \rangle \in T \),
  hence the claim follows from \( v \in T^{\vee} \).
\end{proof}

\begin{proposition}
  \label{polyhedral-of-fg}
  \lean{PointedCone.IsPolyhedral.of_fg}
  \leanok
  If \( \sigma \in N \) is finitely generated, then it is polyhedral.
\end{proposition}
\begin{proof}
  \uses{bot-polyhedral, fourier-motzkin}
  \leanok
  Write \( \sigma = \Cone(S) \) and use induction on the size of \( S
  \): For \( S = \emptyset \), we use~\ref{bot-polyhedral}
  and the induction step is~\ref{fourier-motzkin}.
\end{proof}

\begin{proposition}
  \label{double-dual-finite-set}
  \uses{}
  \lean{PointedCone.dual_dual_eq_span}
  \leanok
  For a finite set \( S \subseteq N \), we have \( S^{\vee\vee} =
  \Cone(S) \).
\end{proposition}
\begin{proof}
  \uses{polyhedral-of-fg, 1-2-1-double-dual-polyhedral-cone}
  \leanok
  The cone \( \sigma := \Cone(S) \) is finitely generated, hence
  polyhedral by~\ref{polyhedral-of-fg}. Furthermore, we have \(
  \sigma^{\vee\vee} = S^{\vee\vee} \). Now
  apply~\ref{double-dual-polyhedral-cone}.
\end{proof}

\begin{proposition}[Polyhedral = Finitely generated]
  \label{polyhedral-iff-fg}
  \uses{}
  \lean{PointedCone.isPolyhedral_iff_fg}
  \leanok
  A pointed cone \( \sigma \subseteq N \) is finitely generated iff it is
  polyhedral.
\end{proposition}
\begin{proof}
  \uses{polyhedral-of-fg}
  \leanok
  The implication ``\( \Rightarrow \)'' is
  Proposition~\ref{polyhedral-of-fg}. For the converse, assume
  \( \sigma = S^{\vee} \) for some finite \( S \subseteq M \). Then \(
  \Cone(S) \) is finitely generated, hence polyhedral
  by~\ref{polyhedral-of-fg}. Thus we find a finite \( T
  \subseteq N \) with \( \Cone(S) = T^\vee \). We obtain
  \[
      \sigma = S^{\vee} = \Cone(S)^\vee = T^{\vee\vee} = \Cone(T),
  \]
  where the last equality is due
  to~\ref{double-dual-finite-set}. Thus \( \sigma \) is finitely
  generated.
\end{proof}

\begin{proposition}
  \label{dual-polyhedral-cone}
  \uses{polyhedral-cone}
  \lean{PointedCone.IsPolyhedral.dual}
  \leanok
  If \( \sigma \subseteq N \) is
  polyhedral, then \( \sigma^\vee \subseteq M \) is polyhedral too.
\end{proposition}
\begin{proof}
  \uses{dual-polyhedral, polyhedral-iff-fg}
  \leanok
  \ref{dual-polyhedral} \( + \) \ref{polyhedral-iff-fg}.
\end{proof}


\subsection{Faces and the Face Lattice}

We define faces of polyhedral cones. The main goal is the partial
order relation on the set of faces and showing that there is a
bijection between faces of \( \sigma \) and faces of \( \sigma^{\vee}
\).

\begin{definition}
  \label{perp}
  \uses{}
  \lean{PointedCone.perp}
  \leanok
  For any subset \( A \subseteq M \), we define the \emph{perp space}
  \[
      A^{\perp}\ :=\ \{v \in N\ \mid\ \forall u \in A, \langle v, u \rangle = 0\},
  \]
  which is a pointed cone in \( N \). We also write \( u^{\perp} :=
  \{u\}^{\perp} \).
\end{definition}

\begin{lemma}
  \label{perp-closed-negation}
  \uses{}
  % \lean{}
  % \leanok
  If \( A \subseteq M \) is closed under negation (for instance, \( A
  \subseteq M \) a \( \Bbbk \)-submodule), then \( A^{\perp} =
  A^{\vee} \).
\end{lemma}
\begin{proof}
  \uses{}
  % \leanok
  \( A^{\perp} \subseteq A^{\vee} \) is clear. Let \( x \in A^{\vee}
  \) and \( a \in A \). then \( \langle x, a \rangle \geq 0 \) and \(
  -\langle x, a \rangle = \langle x, -a \rangle \geq 0 \). Hence \(
  \langle x, a \rangle = 0 \) and \( x \in A^{\perp} \).
\end{proof}


\begin{definition}[Face of a cone]
  \label{face}
  \uses{perp}
  \lean{PointedCone.IsFace}
  \leanok
  A \emph{face} of a pointed cone \( \sigma \subseteq N \) is the
  intersection of \( \sigma \) with some hyperplane \( u^{\perp} \)
  for \( u \in \sigma^{\vee} \). We write \( \tau \preceq \sigma \) if
  \( \tau \) is a face of \( \sigma \) and \( \tau \prec \sigma \) if
  \( \tau \) is a \emph{proper} face, i.e. \( \tau \preceq \sigma \)
  and \( \tau \neq \sigma \).
\end{definition}

\begin{proposition}
  \label{cone-inter-perp}
  \uses{perp, dual-cone}
  % \lean{}
  % \leanok
  For any subset \( S \subseteq N \) and \( u \in S^{\vee} \), we have
  \[
      \Cone(S) \cap u^{\perp} = \Cone(S \cap u^{\perp}).
  \]
\end{proposition}
\begin{proof}
    Both inclusions should be a simple Span-induction.
\end{proof}

\begin{lemma}
  \label{face-polyhedral}
  \uses{face, polyhedral-cone}
  \lean{PointedCone.face_polyhedral}
  \leanok
  A face of a polyhedral cone is polyhedral.
\end{lemma}
\begin{proof}
  \uses{cone-inter-perp}
  % \leanok
  Use \ref{cone-inter-perp} and the fact that if \( S \) is finite,
  so is \( S \cap u^{\perp} \).
\end{proof}



\begin{lemma}
  \label{faces-polyhedral-cone-finite}
  \uses{face-polyhedral}
  % \lean{}
  % \leanok
  A polyhedral cone has only finitely many faces.
\end{lemma}
\begin{proof}
  \uses{}
  % \leanok
  By \ref{face-polyhedral}, a face of a polyhedral cone \( \sigma =
  \Cone(S) \) is of the form \( \Cone(S') \) for some \( S' \subseteq
  S \). Being finite, \( S \) has only finitely many subsets, which
  shows the claim.
\end{proof}


\begin{lemma}[Intersection of faces]
  \label{face-intersection}
  \uses{face}
  \lean{PointedCone.face_intersection}
  \leanok
  If \( \sigma \) is a pointed cone, then the intersection of two
  faces of \( \sigma \) is a again face of \( \sigma \).
\end{lemma}
\begin{proof}
  \uses{}
  % \leanok
  Let \( \tau = \sigma \cap u^{\perp} \) and \( \tau' = \sigma \cap
  \tau'^{\perp} \) be faces. We claim that \( \sigma \cap u^{\perp}
  \cap u'^{\perp} = \sigma \cap (u + u')^{\perp} \). The inclusion
  ``\( \subseteq \)'' is obvious. For the converse, suppose \( v \in
  \sigma \) with \( \langle v, u \rangle + \langle v, u' \rangle = 0
  \). Since \( u, u' \in \sigma^{\vee} \), both summands are
  nonnegative, hence \( \langle v, u \rangle = \langle v, u' \rangle =
  0 \).
\end{proof}

\begin{lemma}
  \label{face-dual-eq-sum}
  \uses{face}
  % \lean{}
  % \leanok
  Let \( \tau \preceq \sigma \) be a face of a pointed cone. Then \(
  \tau^{\vee} = \tau^{\perp} + \sigma^{\vee} \).
\end{lemma}
\begin{proof}
  \uses{perp-closed-negation}
  % \leanok
  It's easy to show that \( \tau = \Span_{\Bbbk}(\tau) \cap
  \sigma^{\vee} \). Dualising gives \( \tau^{\vee} =
  \Span_{\Bbbk}(\tau)^{\vee} + \sigma^{\vee} \).
  Apply~\ref{perp-closed-negation} to get \(
  \Span_{\Bbbk}(\tau)^{\vee} = \Span_{\Bbbk}(\tau)^{\perp} =
  \tau^{\perp} \).
\end{proof}


% TODO do we need LinearOrder?
\begin{lemma}[Face of a face]
  \label{face-face}
  \uses{face}
  \lean{PointedCone.face_face}
  % \leanok
  A face of a face of a polyhedral cone \( \sigma \) is again a face
  of \( \sigma \).
\end{lemma}
\begin{proof}
  \uses{face-dual-eq-sum}
  % \leanok
  Let \( \tau = \sigma \cap u^{\perp} \) with \( u \in \sigma^{\vee}
  \) and \( \rho = \tau \cap v^{\perp} \) with \( v \in \tau^{\vee}
  \). By \ref{face-dual-eq-sum}, we can write \( v = v_1 + v_2 \)
  with \( v_1 \in \tau^{\perp} \) and \( v_2 \in \sigma^{\vee} \).
  Then \( u + v_2 \in \sigma^{\vee} \) and we claim \( \rho = \sigma
  \cap (u + v_2)^{\perp} \). For ``\( \subseteq \)'', let \( x \in
  \rho = \tau \cap v^{\perp} \). Since \( x \in \tau \), we get \(
  \langle x, u \rangle = 0 \) and \( \langle x, v_1 \rangle = 0 \).
  Since \( x \in v^{\perp} \), also \( \langle x, v \rangle = 0 \),
  which by \( v = v_1 + v_2 \) implies \( \langle x, v_2 \rangle = 0
  \). In total, \( \langle x, u + v_2 \rangle = 0 \). For the
  converse, assume \( x \in \sigma \cap (u + v_2)^{\perp} \), i.e. \(
  \langle x, u \rangle + \langle x, v_2 \rangle = 0 \). Since both \(
  u, v_2 \in \sigma^{\vee} \) and \( x \in \sigma \), both summands
  are nonnegative, hence \( \langle x, u \rangle = \langle x, v_2
  \rangle = 0 \). This implies \( x \in \tau \), thus also \( \langle
  x, v_1 \rangle = 0 \). In total, \( \langle x, v \rangle = \langle
  x, v_1 \rangle + \langle x, v_2 \rangle = 0 \), thus \( x \in \rho
  \).
\end{proof}

In particular, Lemma~\ref{face-face} establishes a partial order on
the set of faces of a polyhedral cone \( \sigma \) (written \(
\faces(\sigma) \)). It is finite by
~\ref{faces-polyhedral-cone-finite}.

\begin{lemma}[Face order is a lattice]
  \label{face-lattice}
  \uses{}
  % \lean{}
  % \leanok
  Given a polyhedral cone, the partial order on its faces
  is a lattice. We call it the \emph{face lattice} of the cone.
\end{lemma}
\begin{proof}
Top is the entire cone, bot is the empty cone.
\end{proof}

\begin{lemma}[Face lattice is graded]
  \label{face-lattice-graded}
  \uses{face-lattice}
  % \lean{}
  % \leanok
  Given a polyhedral cone, the partial order on its faces
  is graded.
\end{lemma}
\begin{proof}
The grading is given by the number of faces of this face.
\end{proof}

\begin{definition}[Combinatorial Equivalence]
  \label{combinatorial-equivalence}
  \uses{face-lattice}
  % \lean{}
  % \leanok
  Two polyhedral cones are \emph{combinatorially equivalent} if there
  is an order isomorphism between their face lattices.
\end{definition}

\begin{lemma}[Intervals are face lattices]
  \label{face-lattice-intervals}
  \uses{face-lattice}
  % \lean{}
  % \leanok
  For each interval of the face lattice of a polyhedral cone, there exists
  a polyhedral cone such that the interval is isomorphic to its face lattice.
\end{lemma}
\begin{proof}
\end{proof}

\subsection{Polarity}

\begin{definition}[Polar Cone]
  \label{polar-cone}
  \uses{pointed-cone}
  % \lean{}
  % \leanok
  For a pointed cone \( \sigma \), the {\bf polar cone of \( \sigma \)} is
  the pointed cone
  \[
      S^\circ := \{x \in \sigma \mid \forall y \in \sigma, \langle x, y \rangle
      \leq 0 \}
  \]
\end{definition}

% \subsection{Relative Interior}
% Next, we need the notion of relative interior, which we define here
% completely algebraically. When working over \( \R \), it is the same
% as the topological interior of \( \sigma \) in its span.

% \begin{definition}[Relative interior]
%   \label{rel-interior}
%   \uses{dual-cone, perp}
%   % \lean{}
%   % \leanok
%   The \emph{relative interior} of a pointed cone \( \sigma \subseteq N
%   \) (written \( \sigma^{\circ} \)) is the subset of all \( v \in
%   \sigma \) such that \( \langle v, u \rangle > 0 \) for all \( u \in
%   \sigma^{\vee} \backslash \sigma^{\perp} \). Note that \(
%   \sigma^{\circ} \) is not a pointed cone (it does not contain \( 0
%   \)), but it is closed under addition and \( \Bbbk_{\geq 0} \)-scalar
%   multiplication.
% \end{definition}

% \begin{lemma}
%   \label{rel-interior-perp}
%   \uses{}
%   % \lean{}
%   % \leanok
%   For a pointed cone \( \sigma \), we have \( v \in \sigma^{\circ} \)
%   if and only if \( \sigma^{\perp} = \sigma^{\vee} \cap v^{\perp} \).
% \end{lemma}
% \begin{proof}
%   \uses{}
%   % \leanok
%   Easy, just reformulating the definition.
% \end{proof}

% \begin{lemma}
%   \label{rel-interior-positive-lincomb}
%   \uses{rel-interior}
%   % \lean{}
%   % \leanok
%   Let \( \sigma = \Cone(S) \) be a polyhedral cone with \( S = \{v_1,
%   \dots, v_n\} \). Then \( v \in \sigma^{\perp} \) iff \( v =
%   \sum_{i=1}^n \lambda_i v_i \) for positive scalars \( \lambda_i > 0
%   \).
% \end{lemma}
% \begin{proof}
%   \uses{}
%   % \leanok
%   One direction is easy: Suppose \( v = \sum_{i=1}^n \lambda_i v_i \)
%   for \( \lambda_i > 0 \). We claim \( v \in \sigma^{\circ} \). Let \(
%   u \in \sigma^{\vee} \backslash \sigma^{\perp} \) be arbitrary. Since
%   \( u \notin \sigma^{\perp} \), there exists an \( i \) with \(
%   \langle v_i, u \rangle > 0 \). This implies
%   \[
%       \langle v, u \rangle = \sum_{i=1}^n \lambda_i \langle v_i, u
%       \rangle > 0.
%   \]
%   The other direction is a bit more involved, TODO.
% \end{proof}

% \begin{lemma}
%   \label{rel-interior-nonempty}
%   \uses{}
%   % \lean{}
%   % \leanok
%   The relative interior of a polyhedral cone is nonempty.
% \end{lemma}
% \begin{proof}
%   \uses{rel-interior-positive-lincomb}
%   % \leanok
%   Follows directly from~\ref{rel-interior-positive-lincomb}
% \end{proof}

% \begin{definition}[Dual face]
%   \label{dual-face}
%   \uses{dual-cone, face}
%   % \lean{}
%   % \leanok
%   Given a cone \( \sigma \) and a face \( \tau \preceq \sigma \), the
%   \emph{dual face} to \( \tau \) is
%   \[
%       \tau^* := \sigma^{\vee} \cap \tau^{\perp}.
%   \]
% \end{definition}

% The dual face is actually a face:

% \begin{proposition}
%   \label{dual-face-face-dual}
%   \uses{dual-face, rel-interior-nonempty}
%   % \lean{}
%   % \leanok
%   If \( \tau \preceq \sigma \), then \( \tau^* \preceq \sigma^\vee \).
% \end{proposition}
% \begin{proof}
%   \uses{rel-interior-nonempty, double-dual-polyhedral-cone,
%   rel-interior-perp}
%   % \leanok
%   Let \( \tau = \sigma \cap u^{\perp} \) for \( u \in \sigma^{\vee}
%   \). We need to find a \( v \in \sigma^{\vee\vee} = \sigma \) such
%   that \( \tau^* = \sigma \cap v^{\perp} \).
%   By~\ref{rel-interior-nonempty}, we may pick \( v \in
%   \sigma^{\circ} \), which works by~\ref{rel-interior-perp}.
% \end{proof}


% \begin{proposition}[The dual of a face is antitone]
%   \label{dual-face-antitone}
%   \uses{dual-face}
%   % \lean{}
%   % \leanok
%   For two faces \( \tau_1, \tau_2 \preceq \sigma \) of a polyhedral
%   cone \( \sigma \), we have \( \tau_1 \preceq \tau_2 \) if and only
%   if \( \tau_2^* \preceq \tau_1^* \).
% \end{proposition}
% \begin{proof}
%   \uses{}
%   % \leanok
%   TODO (Shouldn't be too hard hopefully, see maybe Propositions
%   A.5/A.6 of \cite{Oda_1988}).
% \end{proof}

% \begin{proposition}
%   \label{double-dual-face-dual-face}
%   \uses{dual-face}
%   % \lean{}
%   % \leanok
%   If $\tau \preceq \sigma$, then $\tau^{**} = \tau$.
% \end{proposition}
% \begin{proof}
%   \uses{1-2-4-double-dual-polyhedral-cone}
%   % \leanok
%   Classic. See \cite{Oda_1988} maybe.
% \end{proof}

% \begin{proposition}
%   \label{faces-galois-connection}
%   \uses{}
%   % \lean{}
%   % \leanok
%   There is a Galois connection \( \faces(\sigma) \cong
%   \faces(\sigma^{\vee})^{op} \) between the partial orders of faces of
%   \( \sigma \) and faces of \( \sigma^{\vee} \).
% \end{proposition}

% In particular, since \( \faces(\sigma) \) has a maximal (top) element,
% namely \( \sigma \) itself, it must also have a minimal (bottom)
% element. This will also be called the lineality space, which we define
% in the next section.



% \subsection{Strong Convexity}

% \emph{Strongly convex cones} are those containing no nontrivial linear
% subspace. Confusingly, they are sometimes called \emph{pointed cones}
% in the literature, a terminology we cannot use because for us, pointed
% cones are just any \( \Bbbk_{\geq} \)-submodules. Another name is
% \emph{salient cones}.

% \begin{definition}
%   \label{lineality}
%   \uses{pointed-cone}
%   % \lean{}
%   % \leanok
%   The \emph{lineality space} of a pointed cone \( \sigma \) is defined
%   as \( \sigma_{\lin} := \sigma \cap (-\sigma) \). Clearly, \(
%   \sigma_{\lin} \) is a \( \Bbbk \)-subvectorspace of \( N \).
% \end{definition}

% \begin{lemma}
%   \label{lineality-eq-bottom}
%   \uses{lineality, faces-galois-connection}
%   % \lean{}
%   % \leanok
%   The lineality space \( \sigma_{\lin} \) is the minimal (bottom)
%   element in the partial order of the faces of \( \sigma \) (In
%   particular, it is a face).
% \end{lemma}
% \begin{proof}
%   \uses{perp-closed-negation}
%   % \leanok
%   By the Galois connection~\ref{faces-galois-connection}, it is
%   enough to show \( \sigma_{\lin}^* = \sigma^{\vee} \) (the top
%   element in \( \faces(\sigma^{\vee}) \)). We have \( \sigma_{\lin}^* =
%   \sigma^{\vee} \cap \sigma_{\lin}^{\perp} \). Hence we need to show
%   \( \sigma^{\vee} \subseteq \sigma_{\lin}^{\perp} \). This follows by
%   applying the dual to \( \sigma_{\lin} \subseteq \sigma \) and
%   using~\ref{perp-closed-negation}.
% \end{proof}

% \begin{definition}
%   \label{strongly-convex}
%   \uses{lineality}
%   % \lean{}
%   % \leanok
%   A pointed cone \( \sigma \) is called \emph{strongly convex}, if \(
%   \sigma_{\lin} = \{0\} \).
% \end{definition}

% \begin{lemma}
%   \label{strongly-convex-iff-zero-face}
%   \uses{lineality-eq-bottom}
%   % \lean{}
%   % \leanok
%   A polyhedral cone \( \sigma \) is strongly convex iff \( \{0\}
%   \preceq \sigma \).
% \end{lemma}
% \begin{proof}
%   \uses{}
%   % \leanok
%   Both directions follow immediately from \ref{lineality-eq-bottom}.
% \end{proof}

% \begin{lemma}
%   \label{subvectorspace-contained-in-faces}
%   \uses{face}
%   % \lean{}
%   % \leanok
%   Let \( V \subseteq N \) be closed under negation (for instance, a \(
%   \Bbbk \)-submodule) with \( V \subseteq \sigma \) for a polyhedral
%   cone \( \sigma \). Then \( V \) is contained in any face of \(
%   \sigma \).
% \end{lemma}
% \begin{proof}
%   \uses{}
%   % \leanok
%   Let \( \tau = \sigma \cap u^{\perp} \) be a face with \( u \in
%   \sigma^{\vee} \) and let \( x \in V \). Since \( V \subseteq \sigma
%   \), we have \( \langle x, u \rangle \geq 0 \). But also \( -x \in V
%   \), hence \( -\langle x, u \rangle \geq 0 \). We get \( \langle x,
%   u \rangle = 0 \).
% \end{proof}

% \begin{lemma}
%   \label{strongly-convex-iff-no-subspace}
%   \uses{}
%   % \lean{}
%   % \leanok
%   A polyhedral cone \( \sigma \) is strongly convex iff it contains no
%   nontrivial \( \Bbbk \)-subvectorspace.
% \end{lemma}
% \begin{proof}
%   \uses{}
%   % \leanok
%   Both directions follow immediately from
%   \ref{subvectorspace-contained-in-faces}.
% \end{proof}


% \subsection{Lattice Cones}

% In this section, \( N \) and \( M \) denote mutually dual lattices (=
% torsion-free finitely generated \( \Z \)-modules). Extending scalars,
% we get vector spaces \( N_{\Bbbk} := N \otimes_{\Z} \Bbbk \) and \(
% M_{\Bbbk} := M \otimes_{\Z} \Bbbk \) which are mutually dual vector
% spaces as before (over any linearly ordered field \( \Bbbk \)). We
% regard the lattices as subsets of the vector spaces, i.e. \( N
% \subseteq N_{\Bbbk} \) and \( M \subseteq M_{\Bbbk} \). The
% \emph{lattice cones} are those polyhedral cones which are generated by
% a subset of the lattice \( S \subseteq N \) (as opposed to an
% arbitrary subset of the vector space). Cox, Little and Schenck
% \cite{Cox_2011} call them~\emph{rational cones}, which might be
% confusing because they can be defined over any field and don't
% necessarily involve the rational numbers. However, if \( \Bbbk = \Q
% \), every polyhedral cone is indeed rational.

% \begin{definition}
%   \label{lattice-cone}
%   \uses{cone-span}
%   % \lean{}
%   % \leanok
%   A polyhedral cone \( \sigma \subseteq N_{\Bbbk} \) is called a
%   \emph{lattice cone}, if \( \sigma = \Cone(S) \) for some finite set
%   \( S \subseteq N \).
% \end{definition}

% \begin{lemma}
%   \label{face-lattice-cone}
%   \uses{face, lattice-cone}
%   % \lean{}
%   % \leanok
%   If \( \tau \preceq \sigma \) is a face of a lattice cone, then
%   \( \tau \) is also a lattice cone.
% \end{lemma}
% \begin{proof}
%   \uses{}
%   % \leanok
%   Follows directly from~\ref{cone-inter-perp}.
% \end{proof}


% \begin{lemma}
%   \label{dual-lattice-cone}
%   \uses{dual-cone, 04-rat-cone}
%   % \lean{}
%   % \leanok
%   A polyhedral cone \( \sigma \) is a lattice cone iff \(
%   \sigma^{\vee} \) is.
% \end{lemma}
% \begin{proof}
%   \uses{}
%   % \leanok
%   This should be provable by going through the generating set \( T \)
%   defined in the proof of \ref{fourier-motzkin} and seeing that if
%   \( S \subseteq M \) is contained in the lattice, so is \( T \).
% \end{proof}

% \begin{proposition}[Gordan's lemma]
%   \label{gordan-lemma}
%   \uses{lattice-cone}
%   % \lean{}
%   % \leanok
%   If \( \sigma \subset N_{\Bbbk} \) is a polyhedral lattice cone, then
%   \( \sigma \cap N \) is finitely generated as a submonoid of \( N \).
% \end{proposition}
% \begin{proof}
%   \uses{}
%   % \leanok
%   TODO (Classic).
% \end{proof}

% \begin{definition}
%   \label{ray-gen}
%   \uses{edge, lattice-cone}
%   % \lean{}
%   % \leanok
%   If \( \rho \) is an edge of a lattice cone $\sigma$, then the
%   monoid $\rho \cap N$ is generated by a unique element $u_\rho \in
%   \rho \cap N$, which we call the \emph{ray generator} of $\rho$.
% \end{definition}


% \begin{definition}
%   \label{min-gen}
%   \uses{ray-gen}
%   % \lean{}
%   % \leanok
%   The \emph{minimal generators} of a lattice cone $\sigma$ are the
%   ray generators of its edges.
% \end{definition}


% \begin{lemma}
%   \label{min-gen-generate}
%   \uses{strongly-convex, min-gen}
%   % \lean{}
%   % \leanok
%   A strongly convex polyhedral lattice polyhedral cone is generated by
%   its minimal generators.
% \end{lemma}
% \begin{proof}
%   \uses{}
%   % \leanok
%   TODO.
% \end{proof}


% \begin{definition}
%   \label{regular-cone}
%   \uses{min-gen}
%   % \lean{}
%   % \leanok
%   A strongly convex polyhedral lattice cone $\sigma$ is \emph{regular}
%   (also called \emph{smooth}), if its minimal generators form part of
%   a $\Z$-basis of $N$.
% \end{definition}


% \begin{definition}
%   \label{simplicial-cone}
%   \uses{min-gen}
%   % \lean{}
%   % \leanok
%   A strongly convex polyhedral lattice cone $\sigma$ is
%   \emph{simplicial} if its minimal generators are $\R$-linearly
%   independent.
% \end{definition}


% \subsection{Lattice Fans}

% TODO: Define lattice fans, smooth and simplicial, support, ray
% generators, face fan of a cone.

